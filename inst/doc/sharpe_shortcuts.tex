
\typeout{-- sharpe_shortcuts.tex}

%\usepackage[environments,commands,meshstuff,shortcuts]{sepmath}
%\usepackage[environments,commands,shortcuts]{sepmath}

\RequirePackage{ifthen}
\RequirePackage{xspace}

% ack.
%\providecommand{\figref}[1]{Figure~\ref{fig:#1}}
\providecommand{\figref}[1]{Figure\nobreakspace\ref{fig:#1}}
\providecommand{\eqnref}[1]{Equation\nobreakspace\ref{eqn:#1}}

%%%%%%%%%%%%%%%%%%%%%%%%%%%%%%%%%%%%%%%%%%%%%%%%%%%%%%%%%%%%%%%%%%%%%%%%
% meta meta commands
% emptyP if 1 is empty give 2 else give 3
%\providecommand{\mtP}[3]{\ifx\@empty#1\@empty#2\else#3\fi}
%\def\mtP#1#2#3{\ifx\@empty#1\@empty#2\else#3\fi}
%\def\mtP#1#2#3{\ifx\@empty\detokenize{#1}\@empty#2\else#3\fi}
\def\mtP#1#2#3{\if\relax\detokenize{#1}\relax#2\else#3\fi}
% listmore if 1 is empty give 1 else give `,1'
%\def\lMr#1{\ifx\@empty#1\@empty\relax\else{,#1}\fi}
\def\lMr#1{\ifx\@empty\detokenize{#1}\@empty\relax\else{,#1}\fi}

\providecommand{\MATHIT}[1]{\ensuremath{#1}\xspace}
\providecommand{\neUL}[3]{\mtP{#2}{\mtP{#3}{#1}{{#1}_{#3}}}{\mtP{#3}{{#1}^{#2}}{{#1}^{#2}_{#3}}}}
\providecommand{\neSUP}[2]{\mtP{#2}{#1}{{{#1}^{#2}}}}
\providecommand{\mathSUB}[2]{\MATHIT{\neSUB{#1}{#2}}}

\providecommand{\wrapParens}[1]{\left(#1\right)}
\providecommand{\wrapBraces}[1]{\left\{#1\right\}}
\providecommand{\wrapBracks}[1]{\left[#1\right]}

%\providecommand{\wrapNeParens}[1]{\if\relax\detokenize{#1}\relax\else\wrapParens{#1}\fi}

\providecommand{\wrapNeParens}[1]{\mtP{#1}{}{\wrapParens{#1}}}
\providecommand{\wrapNeBraces}[1]{\mtP{#1}{}{\wrapBraces{#1}}}
\providecommand{\wrapNeBracks}[1]{\mtP{#1}{}{\wrapBracks{#1}}}
\providecommand{\neSUB}[2]{\mtP{#2}{#1}{{{#1}_{#2}}}}

\providecommand{\abs}[1]{\ensuremath{\left| #1 \right|}}
\providecommand{\mathSUB}[2]{\MATHIT{\neSUB{#1}{#2}}}
\renewcommand{\Pr}[1]{\ensuremath{\operatorname{Pr}\left\{#1\right\}}}
\providecommand{\vect}[1]{\MATHIT{\boldsymbol{#1}}}
\providecommand{\eye}{\ensuremath{I}}
\providecommand{\Mtx}[1]{\MATHIT{\mathsf{#1}}}
\providecommand{\MtxUL}[3]{\mathUL{\Mtx{#1}}{#2}{#3}}
\providecommand{\nePAIR}[2]{#1\lMr{#2}}
\providecommand{\VAR}[1]{\ensuremath{\operatorname{var}\wrapNeParens{#1}}}
\providecommand{\mathUL}[3]{\MATHIT{\neUL{#1}{#2}{#3}}}
\providecommand{\mathSUP}[2]{\MATHIT{\neSUP{#1}{#2}}}
\providecommand{\vectUL}[3]{\mathUL{\vect{#1}}{#2}{#3}}
\providecommand{\SEPbbb}[1]{\mathbb{#1}}
\providecommand{\reals}[1]{\ensuremath{\SEPbbb{R}^{#1}}}

\providecommand{\oneby}[1]{\MATHIT{\frac{1}{#1}}}

% yes, I am lazy.
\providecommand{\txtSR}{Sharpe ratio\xspace}

\providecommand{\eg}{\emph{e.g.},\xspace}
\providecommand{\ie}{\emph{i.e.},\xspace}
\providecommand{\nb}{\emph{n.b.},\xspace}
\providecommand{\iid}{\emph{i.i.d.}\xspace}
\providecommand{\viz}{\emph{viz.}\xspace}
\providecommand{\etc}{\emph{etc.}\xspace}
\providecommand{\etal}{\emph{et al.}\xspace}
\providecommand{\cf}{\emph{cf.}\xspace}

\providecommand{\setwo}[2]{\ensuremath{\left\{ #1 \left|\; {#2} \right.\right\}}\xspace}

\providecommand{\defeq}{=_{\mbox{df}}}


% utilities%FOLDUP
% convert something into a function
\providecommand{\funcit}[2]{\MATHIT{#1\wrapNeParens{#2}}}
%compact fraction;
\providecommand{\fracc}[2]{\MATHIT{#1 / #2}}
% with paren wrap
\providecommand{\fraccp}[2]{\MATHIT{\wrapNeParens{#1}/\wrapNeParens{#2}}}
%small 'mbox'
%http://stackoverflow.com/questions/1239786/latex-math-mode-and-mbox-mode
\providecommand{\smbox}[1]{\mbox{\scriptsize #1}}

%\providecommand{\argmax}{\MATHIT{\mbox{argmax}}}
\providecommand{\argmax}{\MATHIT{\mathop{\mathrm{argmax}}}}


%UNFOLD

% all commands%FOLDUP
% vector operator
\providecommand{\vecop}[1]{\funcit{\mbox{vec}}{#1}}
\providecommand{\trace}[1]{\funcit{\mbox{tr}}{#1}}
% is this replicated elsewhere?
\providecommand{\det}[1]{\abs{#1}}
% idealized and conditional SNR functions
\providecommand{\pSNRfoo}[2]{\funcit{\mathSUB{\mbox{SNR}}{#1}}{#2}}
\providecommand{\sSNRfoo}[2]{\funcit{\mathSUB{\hat{\mbox{SNR}}}{#1}}{#2}}

\providecommand{\pSNR}[1]{\pSNRfoo{}{#1}}
\providecommand{\sSNR}[1]{\sSNRfoo{}{#1}}

\providecommand{\SNRfunc}[1]{\pSNRfoo{}{#1}}
\providecommand{\SNRfunci}[1]{\pSNRfoo{i}{#1}}
\providecommand{\SNRfuncu}[1]{\pSNRfoo{u}{#1}}
\providecommand{\SNRfuncc}[1]{\pSNRfoo{c}{#1}}



% transpose and inverse
\providecommand{\tr}[1]{\mathSUP{#1}{\top}}
\providecommand{\minv}[1]{\mathSUP{#1}{-1}}
\providecommand{\trminv}[1]{\mathSUP{#1}{-\top}}

% these are private to this file%FOLDUP
\providecommand{\prvsymi}{x}
\providecommand{\prvsymj}{y}
\providecommand{\prvsymk}{f}
\providecommand{\prvsyml}{z}
\providecommand{\prvsyme}{v}
\providecommand{\prvsymf}{f}
\providecommand{\prvsymv}{v}
\providecommand{\prvsymu}{u}
% prices:
\providecommand{\prvsymp}{p}
% arithmetic and geometric:
\providecommand{\prvsyma}{r}
\providecommand{\prvsymg}{l}

\providecommand{\prvsymPortfolio}{w}
\providecommand{\prvsymPASSTHROUGH}{W}

\providecommand{\prvsymWilk}{U}
\providecommand{\prvsymHLT}{T}
\providecommand{\prvsymPBT}{P}
\providecommand{\prvsymRLR}{R}
%UNFOLD

% prices/mtm
%\providecommand{\pryt}[1][t]{\mathSUB{p}{#1}}
\providecommand{\pryp}[1][]{\mathSUB{\prvsymp}{#1}}
%\providecommand{\grett}[1][t]{\mathSUB{l}{#1}}
%\providecommand{\arett}[1][t]{\mathSUB{r}{#1}}

\renewcommand{\exp}[1]{\ensuremath{e^{#1}}}
\providecommand{\longexp}[1]{\ensuremath{\operatorname{exp}\wrapNeParens{#1}}}

%scalar returns:
\providecommand{\reti}[1][]{\mathSUB{\prvsymi}{#1}}
\providecommand{\retj}[1][]{\mathSUB{\prvsymj}{#1}}
\providecommand{\retk}[1][]{\mathSUB{\prvsymk}{#1}}
\providecommand{\retl}[1][]{\mathSUB{\prvsyml}{#1}}
\providecommand{\retf}[1][]{\mathSUB{\prvsymf}{#1}}
\providecommand{\retv}[1][]{\mathSUB{\prvsymv}{#1}}
\providecommand{\reta}[1][]{\mathSUB{\prvsyma}{#1}}
\providecommand{\retg}[1][]{\mathSUB{\prvsymg}{#1}}

%vector returns:
\providecommand{\vreti}[1][]{\vectUL{\prvsymi}{}{#1}}
\providecommand{\vretj}[1][]{\vectUL{\prvsymj}{}{#1}}
\providecommand{\vretk}[1][]{\vectUL{\prvsymk}{}{#1}}
\providecommand{\vretl}[1][]{\vectUL{\prvsyml}{}{#1}}
\providecommand{\vretf}[1][]{\vectUL{\prvsymf}{}{#1}}
\providecommand{\vretv}[1][]{\vectUL{\prvsymv}{}{#1}}
\providecommand{\vreta}[1][]{\vectUL{\prvsyma}{}{#1}}
\providecommand{\vretg}[1][]{\vectUL{\prvsymg}{}{#1}}

% transpose of same
\providecommand{\trvreti}[1][]{\vectUL{\prvsymi}{\top}{#1}}
\providecommand{\trvretj}[1][]{\vectUL{\prvsymj}{\top}{#1}}
\providecommand{\trvretk}[1][]{\vectUL{\prvsymk}{\top}{#1}}
\providecommand{\trvretl}[1][]{\vectUL{\prvsyml}{\top}{#1}}
\providecommand{\trvretf}[1][]{\vectUL{\prvsymf}{\top}{#1}}
\providecommand{\trvretv}[1][]{\vectUL{\prvsymv}{\top}{#1}}
\providecommand{\trvreta}[1][]{\vectUL{\prvsyma}{\top}{#1}}
\providecommand{\trvretg}[1][]{\vectUL{\prvsymg}{\top}{#1}}

%matrix returns:
\providecommand{\mreti}[1][]{\MtxUL{\MakeUppercase{\prvsymi}}{}{#1}}
\providecommand{\mretj}[1][]{\MtxUL{\MakeUppercase{\prvsymj}}{}{#1}}
\providecommand{\mretk}[1][]{\MtxUL{\MakeUppercase{\prvsymk}}{}{#1}}
\providecommand{\mretl}[1][]{\MtxUL{\MakeUppercase{\prvsyml}}{}{#1}}
\providecommand{\mretf}[1][]{\MtxUL{\MakeUppercase{\prvsymf}}{}{#1}}
\providecommand{\mretv}[1][]{\MtxUL{\MakeUppercase{\prvsymv}}{}{#1}}
\providecommand{\mreta}[1][]{\MtxUL{\MakeUppercase{\prvsyma}}{}{#1}}
\providecommand{\mretg}[1][]{\MtxUL{\MakeUppercase{\prvsymg}}{}{#1}}

% the factor or signal
\providecommand{\sfact}[1][t]{\mathSUB{\prvsymf}{#1}}
\providecommand{\vfact}[1][t]{\vectUL{\prvsymf}{}{#1}}
\providecommand{\trvfact}[1][t]{\vectUL{\prvsymf}{\top}{#1}}
\providecommand{\mfact}[1][]{\MtxUL{\MakeUppercase{\prvsymf}}{}{#1}}

% distribution of factor
\providecommand{\pfacmu}{\vectUL{\mu}{}{f}}
\providecommand{\trpfacmu}{\vectUL{\mu}{\top}{f}}
\providecommand{\pfacsig}{\MtxUL{\Gamma}{}{f}}
\providecommand{\pfacgram}[1][]{\MATHIT{\pfacsig + #1\pfacmu\trpfacmu}}

\providecommand{\sfacmu}{\vectUL{\hat{\mu}}{}{f}}
\providecommand{\trsfacmu}{\vectUL{\hat{\mu}}{\top}{f}}
\providecommand{\sfacsig}{\MtxUL{\hat{\Gamma}}{}{f}}

% the error or residual
\providecommand{\serrt}[1][t]{\mathSUB{\prvsymu}{#1}}
\providecommand{\verrt}[1][t]{\vect{\mathSUB{\prvsymu}{#1}}}
\providecommand{\merrt}[1][]{\MtxUL{\MakeUppercase{\prvsymu}}{}{#1}}

% population and sample versions
\providecommand{\pserrt}[1][t]{\mathSUB{\epsilon}{#1}}
\providecommand{\pverrt}[1][t]{\vect{\mathSUB{\epsilon}{#1}}}
\providecommand{\pmerrt}[1][]{\MtxUL{E}{}{#1}}
\providecommand{\sserrt}[1][t]{\mathSUB{\hat{\epsilon}}{#1}}
\providecommand{\sverrt}[1][t]{\vect{\mathSUB{\hat{\epsilon}}{#1}}}
\providecommand{\smerrt}[1][]{\MtxUL{\hat{E}}{}{#1}}

% population and sample (conditional) markowitz weights and passthrough
\providecommand{\pmarkow}[1][]{\mathSUB{w}{#1}}
\providecommand{\smarkow}[1][]{\mathSUB{\hat{w}}{#1}}
\providecommand{\ppasthru}[1][]{\mathSUB{\prvsymPASSTHROUGH}{#1}}
\providecommand{\spasthru}[1][]{\mathSUB{\hat{\prvsymPASSTHROUGH}}{#1}}
\providecommand{\trppasthru}[1][]{\mathUL{\prvsymPASSTHROUGH}{\top}{#1}}
\providecommand{\trspasthru}[1][]{\mathUL{\hat{\prvsymPASSTHROUGH}}{\top}{#1}}

\providecommand{\apasthru}[1][a]{\ppasthru[#1]}
\providecommand{\trapasthru}[1][a]{\trppasthru[#1]}

\providecommand{\ppasopt}[1][]{\ppasthru[\nePAIR{*}{#1}]}
\providecommand{\spasopt}[1][]{\spasthru[\nePAIR{*}{#1}]}

% HE eigenvalues
\providecommand{\pheeig}[1][]{\mathUL{\lambda}{}{#1}}
\providecommand{\sheeig}[1][]{\mathUL{\hat{\lambda}}{}{#1}}

% MGLH
\providecommand{\pWILK}[1][]{\mathUL{\prvsymWilk}{}{#1}}
\providecommand{\pHLT}[1][]{\mathUL{\prvsymHLT}{}{#1}}
\providecommand{\pPBT}[1][]{\mathUL{\prvsymPBT}{}{#1}}
\providecommand{\pRLR}[1][]{\mathUL{\prvsymRLR}{}{#1}}
\providecommand{\sWILK}[1][]{\mathUL{\hat{\prvsymWilk}}{}{#1}}
\providecommand{\sHLT}[1][]{\mathUL{\hat{\prvsymHLT}}{}{#1}}
\providecommand{\sPBT}[1][]{\mathUL{\hat{\prvsymPBT}}{}{#1}}
\providecommand{\sRLR}[1][]{\mathUL{\hat{\prvsymRLR}}{}{#1}}



%vector of all ones, zeros
\providecommand{\vone}[1][]{\vect{1_{#1}}}
\providecommand{\vzero}{\vect{0}}

% a vector
\providecommand{\avect}{\vect{v}}

%leverage variable;
\providecommand{\levi}[1][]{\mathSUB{l}{#1}}

%generic combination of matrices
\providecommand{\ABA}[2]{#1 #2 #1}
\providecommand{\ABCBA}[3]{#1 #2 #3 #2 #1}
\providecommand{\trAB}[2]{\tr{#1}#2}

%gram matrix
\providecommand{\gram}[1]{\trAB{#1}{#1}}
%outer gram
\providecommand{\ogram}[1]{#1 \tr{#1}}
%quadratic form
\providecommand{\qform}[2]{\tr{#2} #1 #2}
%quadratic outer form
\providecommand{\qoform}[2]{#2 #1 \tr{#2}}
\providecommand{\crossp}[1]{\gram{\wrapNeParens{#1}}}

% a conditional on b
%\providecommand{\acondb}[2]{\MATHIT{#1\,\left|\,#2\right.}}
\providecommand{\acondb}[2]{\MATHIT{#1\left|\,#2\right.}}

% sample statistics%FOLDUP
\providecommand{\smean}[1]{\MATHIT{\bar{#1}}}
\providecommand{\sstd}[1]{\mathSUB{s}{#1}}

%\providecommand{\svmean}[1]{\smean{\vect{#1}}}
%\providecommand{\svstd}[1]{\MtxUL{S}{}{#1}}

%sample statistics.
%\providecommand{\smu}[1][]{\smean{\reti[#1]}}
%\providecommand{\ssig}[1][]{\MATHIT{s_{#1}}}
%\providecommand{\smu}[1][]{\smean{\reti[#1]}}
\providecommand{\smu}[1][]{\mathSUB{\hat{\mu}}{#1}}
\providecommand{\ssig}[1][]{\mathSUB{\hat{\sigma}}{#1}}
\providecommand{\ssds}[1][]{\MATHIT{{s_N}_{#1}}}

% risk free
\providecommand{\rfr}[1][0]{\mathSUB{r}{#1}}

% the symbol for SNR/SR
%\providecommand{\prvSNR}[0]{\psi}
\providecommand{\prvSNR}[0]{\zeta}

% sample sharpe ratio
\providecommand{\ssrUL}[2]{\mathUL{\hat{\prvSNR}}{#1}{#2}}
%\providecommand{\ssrUL}[2]{\mathUL{r}{#1}{#2}}
\providecommand{\ssr}[1][]{\ssrUL{}{#1}}
\providecommand{\ssrsq}[1][]{\ssrUL{2}{#1}}
\providecommand{\ssropt}{\ssr[*]}
\providecommand{\ssrsqopt}{\ssrsq[*]}

%\providecommand{\svmu}[1][]{\smean{\vreti[#1]}}
%\providecommand{\svsig}[1][]{\MATHIT{S_{#1}}}
\providecommand{\svmu}[1][]{\vectUL{\hat{\mu}}{}{#1}}
\providecommand{\svsig}[1][]{\MtxUL{\hat{\Sigma}}{}{#1}}

\providecommand{\sfvmu}[1][]{\smean{\vreti[#1]^{*}}}
\providecommand{\sfse}{\kappa}

%unbiased estimator of Sharpe ratio
\providecommand{\susr}[1][]{\MATHIT{\tilde{\prvSNR}_{#1}}}
\providecommand{\sgossz}[1][]{\mathSUB{z}{#1}}
%UNFOLD
%population parameters%FOLDUP
\providecommand{\pmu}[1][]{\mathSUB{\mu}{#1}}
\providecommand{\psig}[1][]{\mathSUB{\sigma}{#1}}
\providecommand{\psigsq}[1][]{\mathUL{\sigma}{2}{#1}}

\providecommand{\psrUL}[2]{\mathUL{\prvSNR}{#1}{#2}}

\providecommand{\psr}[1][]{\psrUL{}{#1}}
\providecommand{\psnr}[1][]{\psrUL{}{#1}}
\providecommand{\psrsq}[1][]{\psrUL{2}{#1}}
\providecommand{\psnrsq}[1][]{\psrUL{2}{#1}}

% optimum SNR in the population
\providecommand{\psnropt}{\psnr[*]}
\providecommand{\psnrpopt}{\psnr[*]}
\providecommand{\psnrsqopt}{\psnrsq[*]}
\providecommand{\psnrsqpopt}{\psnrsq[*]}

% population SNR of the *sample optimal* portfolio. bleah
\providecommand{\psnrsopt}{\psnr[s,*]}

%population vector mean and covariance
\providecommand{\pvmu}[1][]{\vectUL{\mu}{}{#1}}
\providecommand{\pvsig}[1][]{\MtxUL{\Sigma}{}{#1}}
\providecommand{\pvschol}[1][]{\MtxUL{C}{}{#1}}

%UNFOLD

% haircut
\providecommand{\hcut}[1][]{\mathSUB{h}{#1}}
\providecommand{\Pmat}{\Mtx{P}}

\providecommand{\smahalo}[1][]{\mathSUB{\hat{d}}{#1}}


% positively proportional to
\providecommand{\ppropto}{\mathSUB{\propto}{+}}

% MGLH
\providecommand{\MGLHA}[1][]{\MtxUL{A}{}{#1}}
\providecommand{\MGLHC}[1][]{\MtxUL{C}{}{#1}}
\providecommand{\MGLHT}[1][]{\MtxUL{\Theta}{}{#1}}
\providecommand{\MGLHrank}{\MATHIT{r}}
\providecommand{\MGLHa}{\MATHIT{a}}
\providecommand{\MGLHc}{\MATHIT{c}}

\providecommand{\MGLHH}[1][]{\MtxUL{H}{}{#1}}
\providecommand{\MGLHE}[1][]{\MtxUL{E}{}{#1}}
\providecommand{\pMGLHH}[1][]{\MtxUL{H}{}{#1}}
\providecommand{\pMGLHE}[1][]{\MtxUL{E}{}{#1}}
\providecommand{\sMGLHH}[1][]{\MtxUL{\hat{H}}{}{#1}}
\providecommand{\sMGLHE}[1][]{\MtxUL{\hat{E}}{}{#1}}

\providecommand{\eye}[1][]{\MtxUL{I}{}{#1}}


%CDF and quantile%FOLDUP

% make a letter into a distribution 'law'
\providecommand{\makelaw}[2]{\MATHIT{#1\wrapNeParens{#2}}}
\providecommand{\FOOcdf}[3]{\funcit{\mathSUB{F}{#1}}{#2;#3}}
\providecommand{\FOOqnt}[3]{\funcit{\mathSUB{#1}{#2}}{#3}}


%t
\providecommand{\tcdf}[2]{\FOOcdf{t}{#1}{#2}}
\providecommand{\tqnt}[2]{\FOOqnt{t}{#1}{#2}}
\providecommand{\tlaw}[1]{\makelaw{t}{#1}}

\providecommand{\nctcdf}[2]{\FOOcdf{t}{#1}{#2}}
\providecommand{\nctqnt}[2]{\FOOqnt{t}{#1}{#2}}
\providecommand{\nctlaw}[1]{\makelaw{t}{#1}}
\providecommand{\nctvar}[1][]{\mathSUB{t}{#1}}

%f
\providecommand{\fcdf}[2]{\FOOcdf{f}{#1}{#2}}
\providecommand{\ncfcdf}[2]{\FOOcdf{f}{#1}{#2}}
\providecommand{\fqnt}[2]{\FOOqnt{f}{#1}{#2}}
\providecommand{\ncfqnt}[2]{\FOOqnt{f}{#1}{#2}}
\providecommand{\flaw}[1]{\makelaw{F}{#1}}
\providecommand{\ncflaw}[1]{\makelaw{F}{#1}}

%chisq
\providecommand{\prvchisq}{\chi^2}
\providecommand{\chisqcdf}[2]{\FOOcdf{\prvchisq}{#1}{#2}}
\providecommand{\chisqqnt}[2]{\FOOqnt{\prvchisq}{#1}{#2}}
\providecommand{\chisqlaw}[1]{\makelaw{\prvchisq}{#1}}

\providecommand{\gausslaw}[1]{\makelaw{\mathcal{N}}{#1}}

%beta
\providecommand{\betacdf}[2]{\MATHIT{F_{\beta}\wrapNeParens{#1;#2}}}
\providecommand{\betaqnt}[2]{\MATHIT{\beta_{#1}\wrapNeParens{#2}}}
\providecommand{\betalaw}[1]{\makelaw{\mathcal{B}}{#1}}

% instantiations of laws with default parameters filled in?
% overkill?
\providecommand{\normdist}[1][0,1]{\gausslaw{#1}}
\providecommand{\tdist}[1][n]{\MATHIT{\mathcal{t}\wrapNeParens{#1}}}
\providecommand{\hotdist}[1][n,p]{\MATHIT{\mathcal{T^2}\wrapNeParens{#1}}}

%density, distribution, quantile of normal distribution
\providecommand{\dnorm}[1][x]{\funcit{\phi}{#1}}
\providecommand{\pnorm}[1][x]{\funcit{\Phi}{#1}}
\providecommand{\qnorm}[1]{\mathSUB{z}{#1}}
%UNFOLD

%statistical whatsits%FOLDUP
\providecommand{\typeI}{\MATHIT{\alpha}}
\providecommand{\typeII}{\MATHIT{\beta}}
\providecommand{\powr}{\MATHIT{1 - \typeII}}
\providecommand{\irate}{\MATHIT{c_0}}
%UNFOLD

% time commands; change the default!?
\providecommand{\yrto}[1]{\mathSUP{\mbox{yr}}{#1}}
\providecommand{\moto}[1]{\mathSUP{\mbox{mo.}}{#1}}
\providecommand{\qto}[1]{\mathSUP{\mbox{Q}}{#1}}
\providecommand{\dayto}[1]{\mathSUP{\mbox{day}}{#1}}

%linear regression: population and sample%FOLDUP
\providecommand{\pregco}[1][]{\mathSUB{\beta}{#1}}
\providecommand{\pregvec}{\vect{\beta}}
\providecommand{\perr}[1][]{\mathSUB{\epsilon}{#1}}
\providecommand{\sregco}[1][]{\mathSUB{\hat{\beta}}{#1}}
\providecommand{\sregvec}{\vect{\hat{\beta}}}
\providecommand{\serr}[1][]{\mathSUB{\hat{\epsilon}}{#1}}

% multivariate regression;
\providecommand{\pRegco}[1][]{\mathSUB{B}{#1}}
\providecommand{\pErr}[1][]{\mathSUB{E}{#1}}
\providecommand{\pErrt}[1][t]{\pErr[#1]}
\providecommand{\sRegco}[1][]{\mathSUB{\hat{B}}{#1}}
\providecommand{\sErr}[1][]{\mathSUB{\hat{E}}{#1}}

%UNFOLD
%
%'contrast' vector and target
\providecommand{\convec}[1][]{\vect{\mathSUB{v}{#1}}}
\providecommand{\contar}[1][]{\MATHIT{c}}

%noncentrality parameters
\providecommand{\nctp}[1][]{\mathSUB{\delta}{#1}}
\providecommand{\ncfp}[1][]{\mathSUB{\delta}{#1}}

\providecommand{\tstat}[1][]{\mathSUB{t}{#1}}
\providecommand{\fstat}[1][]{\mathSUB{f}{#1}}
\providecommand{\Fstat}[1][]{\mathSUB{F}{#1}}
\providecommand{\Tstat}[1][]{\mathUL{T}{2}{#1}}
%\providecommand{\tbias}[1][\ssiz]{\MATHIT{\eta_{#1}}}
\providecommand{\tbias}[1][\ssiz]{\mathSUB{c}{#1}}

% delta hotelling
\providecommand{\DTstat}[1][]{\MATHIT{\Delta\Tstat[#1]}}


\providecommand{\median}[1]{\funcit{\mbox{median}}{#1}}
\providecommand{\Pr}[1]{\funcit{\mbox{\large P}}{#1}}
\providecommand{\E}[1]{\MATHIT{\mbox{\large E}\wrapNeBracks{#1}}}
%\providecommand{\E}[1]{\ensuremath{\operatorname{E}\left[#1\right]}}
\providecommand{\VAR}[1]{\funcit{\mbox{Var}}{#1}}
\providecommand{\GAM}[1]{\MATHIT{\Gamma\wrapNeParens{#1}}}
\providecommand{\skewness}[1]{\funcit{\mbox{skew}}{#1}}
\providecommand{\exkurt}[1]{\funcit{\mbox{ex\,kurtosis}}{#1}}

\providecommand{\sacor}[1][]{\mathSUB{\hat{\nu}}{#1}}
\providecommand{\pacor}[1][]{\mathSUB{\nu}{#1}}
\providecommand{\corcor}{\MATHIT{d}}

% sample size, # of strategies, number of 'latent factors', addon factors?
% bleah. the semantics of this are hosed.

% sample size; for scalar case
\providecommand{\ssiz}[1][]{\mathSUB{n}{#1}}
% number of 'strategies'; or returns. assets.
\providecommand{\nstrat}[1][]{\mathSUB{k}{#1}}

% number of observations. uhoh.
\providecommand{\nobs}[1][]{\mathSUB{n}{#1}}
% number of 'signals'. let's stick with that.
\providecommand{\nfac}[1][]{\mathSUB{f}{#1}}

% this should probably change to \nstrat.
\providecommand{\nlatf}[1][]{\mathSUB{p}{#1}}
\providecommand{\nlatfmo}[1][]{\mathSUB{q}{#1}}
\providecommand{\nlatftot}{\MATHIT{\nlatf+\nlatfmo}}

% # of attribution factors
% this should probably change to \nfac.
\providecommand{\nattf}[1][]{\mathSUB{l}{#1}}

\providecommand{\df}[1][]{\mathSUB{v}{#1}}

%t-power law numerator constant.
%2FIX: this is a throwaway constant.
\providecommand{\tpowc}[1][]{\mathSUB{k}{#1}}

%aspect ratio
\providecommand{\arat}[1][]{\mathSUB{c}{#1}}

\providecommand{\portw}[1][{}]{\vectUL{w}{}{#1}}
\providecommand{\pportw}[1][{}]{\vectUL{\nu}{}{#1}}
\providecommand{\sportw}[1][{}]{\vectUL{\hat{w}}{}{#1}}
\providecommand{\sportwopt}{\sportw[*]}
\providecommand{\pportwopt}{\pportw[*]}


% basis vectors
\providecommand{\basev}[1][]{\vectUL{e}{}{#1}}

% functions
\providecommand{\farcsin}[1]{\funcit{\arcsin}{#1}}
\providecommand{\fsin}[1]{\funcit{\sin}{#1}}
\providecommand{\farctan}[1]{\funcit{\arctan}{#1}}
\providecommand{\ftan}[1]{\funcit{\tan}{#1}}


%UNFOLD

%for vim modeline: (do not edit)
% vim:ts=2:sw=2:tw=79:fdm=marker:fmr=FOLDUP,UNFOLD:cms=%%s:syn=tex:ft=tex:ai:si:cin:nu:fo=croql:cino=p0t0c5(0:
